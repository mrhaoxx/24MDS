\documentclass[UTF8]{ctexart}
\usepackage{geometry, CJKutf8}
\geometry{margin=1.5cm, vmargin={0pt,1cm}}
\setlength{\topmargin}{-1cm}
\setlength{\paperheight}{29.7cm}
\setlength{\textheight}{25.3cm}

\usepackage{amsfonts}
\usepackage{amsmath}
\usepackage{amssymb}
\usepackage{amsthm}
\usepackage{enumerate}
\usepackage{graphicx}
\usepackage{multicol}
\usepackage{fancyhdr}
\usepackage{layout}
\usepackage{listings}
\usepackage{float, caption}
\usepackage{hyperref}

\lstset{
    basicstyle=\ttfamily,
    basewidth=0.5em
}

\newcommand{\dif}{\mathrm{d}}
\newcommand{\avg}[1]{\left\langle #1 \right\rangle}
\newcommand{\difFrac}[2]{\frac{\dif #1}{\dif #2}}
\newcommand{\pdfFrac}[2]{\frac{\partial #1}{\partial #2}}
\newcommand{\OFL}{\mathrm{OFL}}
\newcommand{\UFL}{\mathrm{UFL}}
\newcommand{\fl}{\mathrm{fl}}
\newcommand{\op}{\odot}
\newcommand{\Eabs}{E_{\mathrm{abs}}}
\newcommand{\Erel}{E_{\mathrm{rel}}}

\title{基于LCS思路的最长严格递增子序列算法设计与分析}
\author{作者:郝星星 \& ChatGPT}
\date{\today}

\begin{document}
\maketitle

\section{问题背景与目标}
在很多应用场景(例如数组处理、动态规划练习等)中,我们常常需要找出给定序列中最长的\textbf{严格递增子序列}(Longest Increasing Subsequence,简称LIS)。严格递增子序列要求所选元素的下标严格递增,且所选元素的值也严格递增。  
例如,对于序列$[3,1,5,2,6,4]$,其所有严格递增子序列的一个示例是$[3,5,6]$。本报告将参考\textbf{LCS问题}(Longest Common Subsequence,最长公共子序列)的思路,设计出一个用于寻找给定序列中LIS的算法,并在此基础上对算法进行进一步的优化。

\section{LIS的\texorpdfstring{$O(n^2)$}{O(n2)}算法}
\subsection{设计思路}
参考LCS的动态规划思想:  
\begin{enumerate}
    \item 设有一个长度为$n$的序列$A = [a_1, a_2, \ldots, a_n]$。
    \item 定义一个长度同为$n$的一维数组$\texttt{dp}$,其中$\texttt{dp}[i]$表示\textbf{以$a_i$结尾}的最长严格递增子序列的长度。
    \item 初始时,每个位置都可以是长度为$1$的递增子序列,故$\texttt{dp}[i] \leftarrow 1$。
    \item 对于每个$i$(从左到右),再向左扫描所有$j < i$。若$a_j < a_i$,则表示可以在以$a_j$结尾的子序列后面接上$a_i$形成更长的子序列,所以:
    \[
        \texttt{dp}[i] = \max(\texttt{dp}[i], \texttt{dp}[j] + 1) \quad \text{前提是 } a_j < a_i.
    \]
    \item 最终答案即为$\max\limits_{1 \le i \le n}\texttt{dp}[i]$。
\end{enumerate}

该方法的时间复杂度主要由\textbf{两层循环}所决定,每次扫描需要$O(n)$,共需要做$n$次扫描,故整体时间复杂度为$O(n^2)$。

\subsection{伪代码示例}
\begin{lstlisting}[language=C]
// 输入:长度为 n 的序列 A[1..n]
// 输出:A 的最长严格递增子序列的长度

function LIS_length_O_n2(A[1..n]):
    // Step 1: dp 数组初始化
    let dp[1..n] be an array
    for i from 1 to n:
        dp[i] <- 1

    // Step 2: 动态规划填表
    for i from 1 to n:
        for j from 1 to i-1:
            if A[j] < A[i]:
                dp[i] = max(dp[i], dp[j] + 1)

    // Step 3: 答案为 dp 数组中的最大值
    let ans = 0
    for i from 1 to n:
        ans = max(ans, dp[i])
    return ans
\end{lstlisting}

\subsection{示例演示}
我们以一个直观的例子来展示算法的执行过程。设有序列:
\[
    A = [5, \,2, \,8, \,6, \,3, \,6, \,9, \,7].
\]
依次计算$\texttt{dp}[i]$的过程如下表所示:

\[
\begin{array}{ccccccccc}
\hline
i             & 1 & 2 & 3 & 4 & 5 & 6 & 7 & 8 \\
\hline
A[i]          & 5 & 2 & 8 & 6 & 3 & 6 & 9 & 7 \\
\hline
init~dp[i]     & 1 & 1 & 1 & 1 & 1 & 1 & 1 & 1 \\
\hline
\end{array}
\]
\[
\begin{array}{c|cccccccc}
\text{更新过程} &   &   &   &   &   &   &   &   \\
\text{(1) i=1} & dp[1]=1 & & & & & & & \\
\text{(2) i=2} & dp[2]=1 & & & & & & & \\
   & \text{查看 j=1: } A[1]=5 \not< A[2]=2 \text{ (不满足)} \\
\text{(3) i=3} & dp[3]=1 & & & & & & & \\
   & j=1: A[1]=5 < A[3]=8 \rightarrow dp[3]=\max(1, dp[1]+1)=2 \\
   & j=2: A[2]=2 < A[3]=8 \rightarrow dp[3]=\max(2, dp[2]+1)=2 \\
\text{(4) i=4} & dp[4]=1 & & & & & & & \\
   & j=1: A[1]=5 < A[4]=6 \rightarrow dp[4]=\max(1, dp[1]+1)=2 \\
   & j=2: A[2]=2 < A[4]=6 \rightarrow dp[4]=\max(2, dp[2]+1)=2 \\
   & j=3: A[3]=8 \not< A[4]=6 \text{(不满足)} \\
\text{(5) i=5} & dp[5]=1 & & & & & & & \\
   & j=1: A[1]=5 < A[5]=3 \text{(不满足)} \\
   & j=2: A[2]=2 < A[5]=3 \rightarrow dp[5]=\max(1, dp[2]+1)=2 \\
   & j=3: A[3]=8 \not< A[5]=3 \text{(不满足)} \\
   & j=4: A[4]=6 \not< A[5]=3 \text{(不满足)} \\
\text{(6) i=6} & dp[6]=1 & & & & & & & \\
   & j=1: A[1]=5 < A[6]=6 \rightarrow dp[6]=\max(1, dp[1]+1)=2 \\
   & j=2: A[2]=2 < A[6]=6 \rightarrow dp[6]=\max(2, dp[2]+1)=2 \\
   & j=3: A[3]=8 \not< A[6]=6 \text{(不满足)} \\
   & j=4: A[4]=6 \not< A[6]=6 \text{(不满足, 严格递增需 < )} \\
   & j=5: A[5]=3 < A[6]=6 \rightarrow dp[6]=\max(2, dp[5]+1)=3 \\
\text{(7) i=7} & dp[7]=1 & & & & & & & \\
   & j=1: A[1]=5 < A[7]=9 \rightarrow dp[7]=\max(1, dp[1]+1)=2 \\
   & j=2: A[2]=2 < A[7]=9 \rightarrow dp[7]=\max(2, dp[2]+1)=2 \\
   & j=3: A[3]=8 < A[7]=9 \rightarrow dp[7]=\max(2, dp[3]+1)=3 \\
   & j=4: A[4]=6 < A[7]=9 \rightarrow dp[7]=\max(3, dp[4]+1)=3 \\
   & j=5: A[5]=3 < A[7]=9 \rightarrow dp[7]=\max(3, dp[5]+1)=3 \\
   & j=6: A[6]=6 < A[7]=9 \rightarrow dp[7]=\max(3, dp[6]+1)=4 \\
\text{(8) i=8} & dp[8]=1 & & & & & & & \\
   & j=1: A[1]=5 < A[8]=7 \rightarrow dp[8]=\max(1, dp[1]+1)=2 \\
   & j=2: A[2]=2 < A[8]=7 \rightarrow dp[8]=\max(2, dp[2]+1)=2 \\
   & j=3: A[3]=8 \not< A[8]=7 \text{(不满足)} \\
   & j=4: A[4]=6 < A[8]=7 \rightarrow dp[8]=\max(2, dp[4]+1)=3 \\
   & j=5: A[5]=3 < A[8]=7 \rightarrow dp[8]=\max(3, dp[5]+1)=3 \\
   & j=6: A[6]=6 < A[8]=7 \rightarrow dp[8]=\max(3, dp[6]+1)=4 \\
   & j=7: A[7]=9 \not< A[8]=7 \text{(不满足)} \\
\hline
\end{array}
\]

\[
\begin{array}{|c|cccccccc|}
\hline
final~dp[i]    & 1 & 1 & 2 & 2 & 2 & 3 & 4 & 4 \\
\hline
\end{array}
\]

由上表可见,$\max(dp) = 4$。最长严格递增子序列的长度为$4$。  
相应地,我们能找到若干个长度为$4$的递增子序列,其中一个是$[2,\,3,\,6,\,9]$,另一个是$[2,\,3,\,6,\,7]$(取最后一个元素$7$同样能形成长度$4$)。

\subsection{算法复杂度分析}
从上述过程可知,该算法在最外层对每一个元素$A[i]$进行遍历($O(n)$),在内层对所有$j<i$进行遍历(最多$O(n)$),故总的时间复杂度为$O(n^2)$。在多数通用场景下,该算法足以胜任规模中等(如$n$在几千到几万)的应用。

\section{\texorpdfstring{$O(n \log n)$}{O(n log n)}算法}
如果想进一步提高效率,可借助以下思路将LIS问题在\textbf{平均或最坏情况下}实现到$O(n \log n)$:
\begin{enumerate}
    \item 准备一个辅助数组(有时称为“tail数组”或者“牌顶数组”),用来记录当前长度的递增子序列的最后一个元素的最小可能值。
    \item 从左到右遍历序列,每遇到一个新的元素$a_i$,通过二分查找的方式在辅助数组中找到它应放置的位置,并进行更新。
    \item 辅助数组的长度就是当前找到的\textbf{最长严格递增子序列长度}。
\end{enumerate}

伪代码简单示例如下:
\begin{lstlisting}[language=C]
// 输入:长度为 n 的序列 A[1..n]
// 输出:LIS 的长度,时间复杂度 O(n log n)
function LIS_length_O_nlogn(A[1..n]):
    // tail[k] 表示“所有长度为 k 的递增子序列的末尾元素的最小值”
    create an empty array/list tail
    for i from 1 to n:
        // 在 tail 中寻找第一个 >= A[i] 的位置pos
        // 使用二分查找
        pos = binary_search_first_geq(tail, A[i])

        // 若 pos 等于 tail 长度,表示 A[i] 比 tail 中任何元素都大,直接插到末尾
        if pos == length(tail):
            tail.append(A[i])
        else:
            // 否则,将 tail[pos] 替换为 A[i]
            tail[pos] = A[i]

    return length(tail)
\end{lstlisting}

总体来说,\textbf{tail数组的长度}始终保持“已能找到的最长递增子序列长度”,而对每个新元素$a_i$的二分查找和替换操作只需$O(\log n)$,遍历$n$个元素后,总耗时$O(n \log n)$。

\section{总结}
本报告首先通过一个$O(n^2)$的\textbf{动态规划算法}演示了如何基于LCS(最长公共子序列)问题的动态规划思路来解决最长严格递增子序列(LIS)问题,并在此基础上给出了一个$O(n \log n)$的思路以便在数据规模更大时获得更高的效率。  


\end{document}