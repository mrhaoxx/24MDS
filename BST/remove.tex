\documentclass[UTF8]{ctexart}
\usepackage{geometry, CJKutf8}
\geometry{margin=1.5cm, vmargin={0pt,1cm}}
\setlength{\topmargin}{-1cm}
\setlength{\paperheight}{29.7cm}
\setlength{\textheight}{25.3cm}

% useful packages.
\usepackage{amsfonts}
\usepackage{amsmath}
\usepackage{amssymb}
\usepackage{amsthm}
\usepackage{enumerate}
\usepackage{graphicx}
\usepackage{multicol}
\usepackage{fancyhdr}
\usepackage{layout}
\usepackage{listings}
\usepackage{float, caption}

\lstset{
    basicstyle=\ttfamily, basewidth=0.5em
}

% some common command
\newcommand{\dif}{\mathrm{d}}
\newcommand{\avg}[1]{\left\langle #1 \right\rangle}
\newcommand{\difFrac}[2]{\frac{\dif #1}{\dif #2}}
\newcommand{\pdfFrac}[2]{\frac{\partial #1}{\partial #2}}
\newcommand{\OFL}{\mathrm{OFL}}
\newcommand{\UFL}{\mathrm{UFL}}
\newcommand{\fl}{\mathrm{fl}}
\newcommand{\op}{\odot}
\newcommand{\Eabs}{E_{\mathrm{abs}}}
\newcommand{\Erel}{E_{\mathrm{rel}}}

\title{AVL树的删除实现说明}
\author{郝星星 \& ChatGPT}
\date{\today}
\usepackage{amsmath}
\usepackage{color}
\usepackage{listings}
\usepackage{graphicx}
\usepackage{hyperref}

\begin{document}

\maketitle

\section{AVL树删除实现概述}
AVL树是一种自平衡的二叉搜索树,其特点是每个节点的左右子树高度差不超过1。AVL树的删除操作需要在节点被删除后对树进行平衡,以保证其高度始终尽量保持平衡。本次删除操作的实现为非递归版本,主要的删除步骤包括:

首先,找到待删除的节点及其父节点。通过使用非递归的方式查找节点,避免了递归调用可能导致的栈溢出问题。在找到待删除节点后,根据其子节点的数量进行不同的处理:

\begin{itemize}
    \item 如果待删除节点是叶子节点,直接删除该节点即可。
    \item 如果待删除节点只有一个子节点,则将其子节点提升来替代待删除节点的位置。
    \item 如果待删除节点有两个子节点,则找到右子树中的最小节点,用该节点替代待删除节点,并删除最小节点的位置。
\end{itemize}

在删除节点的过程中,通过使用一个栈来记录从根节点到待删除节点路径上的所有节点。这些节点在删除操作后可能需要重新平衡。非递归查找和删除的优势在于避免了深度较大时的栈溢出风险,同时使代码逻辑更为清晰。

删除节点后,需要根据路径上的节点进行重新平衡。为了实现这一点,栈中的节点会依次被弹出,并针对每个节点进行平衡调整。具体的平衡调整包括左旋、右旋或双旋操作,以确保AVL树的高度差保持在允许范围内。

\section{删除操作的平衡调整}
在AVL树中,每次删除操作后,可能会破坏树的平衡性。因此,在删除节点后需要对路径上的每个节点进行重新平衡。本实现中,通过使用一个栈记录路径上的节点,并在删除后逐个节点进行平衡调整,确保了树的高度差保持在允许范围内。

平衡调整的过程包括:

\begin{itemize}
    \item \textbf{左旋}:当右子树高度过大时进行左旋操作。
    \item \textbf{右旋}:当左子树高度过大时进行右旋操作。
    \item \textbf{双旋}:当某节点的两个子树中高度差过大时,可能需要先左旋再右旋,或者先右旋再左旋,具体取决于失衡的方向。
\end{itemize}

通过这种方式,AVL树在每次删除操作后能够保持平衡,确保查找、插入和删除操作的时间复杂度始终保持在 $O(\log n)$ 的水平。

\end{document}

