\documentclass[UTF8]{ctexart}
\usepackage{geometry, CJKutf8}
\geometry{margin=1.5cm, vmargin={0pt,1cm}}
\setlength{\topmargin}{-1cm}
\setlength{\paperheight}{29.7cm}
\setlength{\textheight}{25.3cm}

% useful packages.
\usepackage{amsfonts}
\usepackage{amsmath}
\usepackage{amssymb}
\usepackage{amsthm}
\usepackage{enumerate}
\usepackage{graphicx}
\usepackage{multicol}
\usepackage{fancyhdr}
\usepackage{layout}
\usepackage{listings}
\usepackage{float, caption}

\lstset{
    basicstyle=\ttfamily, basewidth=0.5em
}

% some common command
\newcommand{\dif}{\mathrm{d}}
\newcommand{\avg}[1]{\left\langle #1 \right\rangle}
\newcommand{\difFrac}[2]{\frac{\dif #1}{\dif #2}}
\newcommand{\pdfFrac}[2]{\frac{\partial #1}{\partial #2}}
\newcommand{\OFL}{\mathrm{OFL}}
\newcommand{\UFL}{\mathrm{UFL}}
\newcommand{\fl}{\mathrm{fl}}
\newcommand{\op}{\odot}
\newcommand{\Eabs}{E_{\mathrm{abs}}}
\newcommand{\Erel}{E_{\mathrm{rel}}}

\title{二叉搜索树(BST)的实现和测试报告}
\author{郝星星 \& ChatGPT}
\date{\today}
\usepackage{amsmath}
\usepackage{color}
\usepackage{listings}
\usepackage{graphicx}
\usepackage{hyperref}

\begin{document}

\maketitle

\section{引言}
本报告记录了二叉搜索树(BST)的实现和测试。测试的目标是验证BST在典型的随机操作以及特定边界情况下的正确性。测试分为两类:随机测试和特殊情况测试,确保一般功能和边界条件都得到了充分的检验。

\section{BST实现概述}
BST的实现支持以下操作:
\begin{itemize}
    \item \textbf{插入}:添加新元素,同时保持BST的特性。
    \item \textbf{删除}:删除元素,处理节点有零个、一个或两个子节点的情况。
    \item \textbf{遍历}:使用中序遍历打印BST中的元素,以保证元素按排序顺序输出。
\end{itemize}
BST实现为模板类,允许存储通用数据类型,并支持中序遍历以验证操作的正确性。

\section{删除函数的修改}
在最新的实现中,删除函数从递归实现修改为非递归实现,以提高删除操作的效率,特别是在深度较大的树上。新的删除函数的代码如下:

\begin{lstlisting}[language=C++]
void remove(const Comparable &x, BinaryNode * &t) {
    // 开始了删除函数的定义,接受待删除的值和指向树根节点的指针引用作为参数。
    if (t == nullptr) {
        // 检查树是否为空。如果为空,则直接返回,因为没有可删除的元素。
        return;
    }

    // 找到节点,使用非递归方式
    BinaryNode *node = t;
    BinaryNode *parent = nullptr;
    // 初始化指针 'node' 指向当前节点,'parent' 指向其父节点,最初设为 nullptr。
    while(node != nullptr){
        if(x < node->element){
            // 如果待删除的值小于当前节点的值,则移动到左子树并更新父节点指针。
            parent = node;
            node = node->left;
        } else if(x > node->element){
            // 如果待删除的值大于当前节点的值,则移动到右子树并更新父节点指针。
            parent = node;
            node = node->right;
        } else {
            // 如果找到了值等于待删除值的节点,则停止查找。
            break;
        }
    }

    if(node == nullptr){
        // 如果未找到待删除的节点,则返回,表示该值不存在于树中。
        return;
    }

    if(node->left != nullptr && node->right != nullptr){
        // 处理有两个子节点的情况,通过分离右子树中的最小节点来替换当前节点。
        BinaryNode *minNode = detachMin(node->right);
        minNode->left = node->left;
        minNode->right = node->right;
        if(parent == nullptr){
            // 如果待删除节点是根节点,则更新根节点指针。
            t = minNode;
        } else {
            if(parent->left == node){
                // 如果待删除节点是父节点的左子节点,则将父节点的左指针更新为新的子节点。
                parent->left = minNode;
            } else {
                parent->right = minNode;
            }
        }
        delete node;
    } else {
        // 处理只有一个子节点的情况,找出非空的子节点。
        BinaryNode *child = node->left != nullptr ? node->left : node->right;
        if(parent == nullptr){
            // 如果待删除节点是根节点,则将根节点更新为其唯一的子节点。
            t = child;
        } else {
            if(parent->left == node){
                parent->left = child;
            } else {
                parent->right = child;
            }
        }
        delete node;
    }
}
\end{lstlisting}

新的实现通过使用循环而不是递归来查找待删除的节点,避免了递归调用可能带来的栈溢出问题。此外,对于节点的删除和重新连接,使用了更加直接的指针操作,从而使得删除过程更加清晰和高效。

在处理具有两个子节点的删除时,新实现通过从右子树中分离出最小节点(\texttt{detachMin} 函数)来替换待删除的节点,从而保持BST的性质。

\section{测试策略}
测试分为两个阶段:随机测试和特殊情况测试。随机测试旨在模拟典型的使用场景,而特殊情况测试则集中于边界条件和特定的挑战性场景。

\subsection{随机测试}
随机测试涉及向BST中插入10个随机生成的元素,以中序遍历打印BST,然后删除5个随机元素。在每次插入和删除后,将BST的结构与维护相同元素的\texttt{std::set}进行比较,以确保其正确性。以下是随机测试的结果:

\begin{verbatim}
Inserting random elements into the BST:
15 98 70 35 81 89 67 35 43 77
Initial tree (in-order):
15 35 43 67 70 77 81 89 98
Elements in set after insertion:
15 35 43 67 70 77 81 89 98

Removing some random elements from the BST:
Trying to remove: 77
The BST and the set match.
Trying to remove: 67
The BST and the set match.
Trying to remove: 98
The BST and the set match.
Trying to remove: 43
The BST and the set match.
Trying to remove: 15
The BST and the set match.
\end{verbatim}

随机测试确认BST能够正确处理插入和删除操作,并保持其特性,正如BST与\texttt{std::set}的输出匹配所证明的那样。

\subsection{特殊情况测试}
特殊情况测试旨在验证BST对边界条件的处理:
\begin{itemize}
    \item \textbf{从空树中删除}:尝试从空BST中删除节点。输出正确地指示树仍然为空。
    \item \textbf{单节点树}:插入一个节点(50)并删除它。BST正确地过渡回空状态。
    \item \textbf{重复插入}:插入重复元素(30)。BST正确地忽略了重复项,并仅保留一个实例。
    \item \textbf{删除叶子节点}:从包含节点15、30和40的树中删除叶子节点15。BST正确地通过删除叶子节点进行了调整。
    \item \textbf{删除只有一个子节点的节点}:删除具有一个子节点(10)的节点。BST正确地保留了子节点(5)。
    \item \textbf{删除有两个子节点的节点}:删除具有两个子节点(25)的节点。BST正确地重组,保持了有效的结构。
\end{itemize}

\subsubsection{特殊情况测试结果}
以下是特殊情况测试的摘要:

\begin{verbatim}
Test 1: Deleting from an empty tree
Empty tree

Test 2: Insert a single node and delete it
Tree after inserting 50:
50
Tree after deleting 50 (should be empty):
Empty tree

Test 3: Insert duplicate elements
Tree after attempting to insert duplicate 30:
30

Test 4: Remove a leaf node
Tree before removing leaf node 15:
15 30 40
Tree after removing leaf node 15:
30 40

Test 5: Remove a node with one child
Tree before removing node 10 (which has one child 5):
5 10 30 40
Tree after removing node 10 (should retain child 5):
5 30 40

Test 6: Remove a node with two children
Tree before removing node 25 (which has two children 20 and 30):
5 20 25 30 40
Tree after removing node 25 (should reorganize correctly):
5 20 30 40
\end{verbatim}

\section{结论}
随机测试和特殊情况测试表明,BST的实现能够在典型和边界情况下正确地管理插入、删除和遍历操作。整个测试过程中,树保持了其结构完整性,并遵循了BST的特性。

最新的删除函数实现通过使用非递归方式,提高了效率并减少了递归调用的栈溢出风险。这使得BST在处理深度较大的树时更加稳健和高效。

未来的工作可以包括扩展实现,以处理其他功能,例如平衡(如AVL树或红黑树),以及实现更全面的性能测试。

\section{参考文献}
\begin{itemize}
    \item Cormen, T. H., Leiserson, C. E., Rivest, R. L., \& Stein, C. (2009). 	\textit{Introduction to Algorithms}. MIT Press.
    \item Sedgewick, R., \& Wayne, K. (2011). 	\textit{Algorithms}. Addison-Wesley.
\end{itemize}

\end{document}
