\documentclass[UTF8]{ctexart}
\usepackage{geometry, CJKutf8}
\geometry{margin=1.5cm, vmargin={0pt,1cm}}
\setlength{\topmargin}{-1cm}
\setlength{\paperheight}{29.7cm}
\setlength{\textheight}{25.3cm}

% useful packages.
\usepackage{amsfonts}
\usepackage{amsmath}
\usepackage{amssymb}
\usepackage{amsthm}
\usepackage{enumerate}
\usepackage{graphicx}
\usepackage{multicol}
\usepackage{fancyhdr}
\usepackage{layout}
\usepackage{listings}
\usepackage{float, caption}

\lstset{
    basicstyle=\ttfamily, basewidth=0.5em
}

% some common command
\newcommand{\dif}{\mathrm{d}}
\newcommand{\avg}[1]{\left\langle #1 \right\rangle}
\newcommand{\difFrac}[2]{\frac{\dif #1}{\dif #2}}
\newcommand{\pdfFrac}[2]{\frac{\partial #1}{\partial #2}}
\newcommand{\OFL}{\mathrm{OFL}}
\newcommand{\UFL}{\mathrm{UFL}}
\newcommand{\fl}{\mathrm{fl}}
\newcommand{\op}{\odot}
\newcommand{\Eabs}{E_{\mathrm{abs}}}
\newcommand{\Erel}{E_{\mathrm{rel}}}

\begin{document}

\pagestyle{fancy}
\fancyhead{}
\lhead{郝星星, 3230101639}
\chead{数据结构与算法第四次作业}
\rhead{Oct.20th, 2024}

\section{测试程序的设计思路}

本次测试的主要思路是对自定义的 List 类进行全面验证,涵盖了各种基本操作、赋值操作、插入与删除、异常处理等功能。我们希望通过这些测试来确认 List 类在不同使用场景下的行为是否符合预期,包括:
\begin{itemize}
\item 默认构造、拷贝构造、移动构造。
\item 插入、删除元素,包括边界情况的测试。
\item 赋值操作、异常情况处理及反向迭代。
\end{itemize}


\subsection{初始化与基本操作}
\begin{itemize}
\item \textbf{默认构造函数测试}:思路是验证空列表是否正确初始化,输出结果为 true 表示初始化成功且列表为空。
\item \textbf{push\_back 和 push\_front 测试}:对列表尾部和头部进行插入操作,插入成功后验证列表大小和内容,以确保操作正确性。
\item \textbf{front 和 back 测试}:思路是获取列表首元素和尾元素,确保插入后的数据与预期一致。
\item \textbf{pop\_front 和 pop\_back 测试}:对头尾元素执行删除操作,删除后检查列表内容及大小,以验证操作的正确性。
\end{itemize}

\subsection{赋值与拷贝测试}
\begin{itemize}
\item \textbf{拷贝赋值运算符测试}:将一个列表的内容拷贝到另一个列表,思路是确保数据在拷贝后的列表中保持一致性。
\item \textbf{移动赋值运算符测试}:移动一个列表的内容到另一个列表,思路是确保原列表在移动后为空且新列表的数据正确。
\end{itemize}

\subsection{插入与删除操作}
\begin{itemize}
\item \textbf{指定位置插入测试}:思路是在列表中间插入元素,验证插入后的数据顺序是否正确。
\item \textbf{插入右值测试}:测试将右值插入到列表中,确保移动语义正常工作,以优化性能。
\item \textbf{删除范围测试}:使用 erase(iterator, iterator) 删除列表中一段元素,确保删除后数据和列表大小符合预期。
\end{itemize}

\subsection{清空与异常处理}
\begin{itemize}
\item \textbf{clear 函数测试}:清空列表,思路是验证列表变为空,以确保清空操作的有效性。
\item \textbf{空列表操作测试}:对空列表执行 pop\_front 和 pop\_back 操作,通过捕获异常来确保程序不崩溃,验证代码的健壮性。
\end{itemize}


\subsection{反向迭代器测试}
\begin{itemize}
\item \textbf{反向迭代器测试}:思路是对列表从尾到头进行遍历,验证反向迭代器的功能是否正常工作,确保双向链表结构的正确性。
\end{itemize}


\section{测试的结果}

在对空列表执行 pop\_front 和 pop\_back 操作时,原始程序会出发segmentation fault,这是因为在 pop\_front 和 pop\_back 函数中没有对空列表进行判断。为了解决这个问题,我在 pop\_front、pop\_back 和 erase 函数中加入了对空列表的判断,当列表为空时,直接抛出异常。


我用 valgrind 进行测试,发现没有发生内存泄露。

\section{bug报告}

我发现了一个 bug,触发条件如下:

\begin{enumerate}
    \item 创建一个空列表;
    \item 对空列表执行 pop\_front 操作;
    \item 程序会出现 segmentation fault。
\end{enumerate}

据我分析,它出现的原因是:

\begin{enumerate}
    \item 在 pop\_front 函数中,没有对空列表进行判断,直接执行了删除操作;
    \item 由于列表为空,删除操作会访问空指针,导致程序崩溃。
\end{enumerate}

\end{document}

%%% Local Variables: 
%%% mode: latex
%%% TeX-master: t
%%% End: 
