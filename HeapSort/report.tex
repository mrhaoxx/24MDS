\documentclass[UTF8]{ctexart}
\usepackage{geometry, CJKutf8}
\geometry{margin=1.5cm, vmargin={0pt,1cm}}
\setlength{\topmargin}{-1cm}
\setlength{\paperheight}{29.7cm}
\setlength{\textheight}{25.3cm}

% useful packages.
\usepackage{amsfonts}
\usepackage{amsmath}
\usepackage{amssymb}
\usepackage{amsthm}
\usepackage{enumerate}
\usepackage{graphicx}
\usepackage{multicol}
\usepackage{fancyhdr}
\usepackage{layout}
\usepackage{listings}
\usepackage{float, caption}

\lstset{
    basicstyle=\ttfamily, basewidth=0.5em
}

% some common command
\newcommand{\dif}{\mathrm{d}}
\newcommand{\avg}[1]{\left\langle #1 \right\rangle}
\newcommand{\difFrac}[2]{\frac{\dif #1}{\dif #2}}
\newcommand{\pdfFrac}[2]{\frac{\partial #1}{\partial #2}}
\newcommand{\OFL}{\mathrm{OFL}}
\newcommand{\UFL}{\mathrm{UFL}}
\newcommand{\fl}{\mathrm{fl}}
\newcommand{\op}{\odot}
\newcommand{\Eabs}{E_{\mathrm{abs}}}
\newcommand{\Erel}{E_{\mathrm{rel}}}

\begin{document}

\pagestyle{fancy}
\fancyhead{}
\lhead{数据结构与算法第七次作业}
\chead{HeapSort 性能分析报告}
\rhead{2024 年 12 月 1 日}

\title{HeapSort with STL}
\author{mrhaoxx \& ChatGPT}

\maketitle

\section{整体设计思路}
本次作业设计了一个基于堆排序的算法,并通过 \texttt{std::make\_heap} 和 \texttt{std::pop\_heap} 实现排序功能。主要功能包括:
\begin{itemize}
    \item 使用标准库的堆操作函数构建堆,排序元素。
    \item 针对随机序列、有序序列、逆序序列、部分重复序列进行性能测试。
\end{itemize}

\section{测试流程设计}
测试程序的主要步骤包括:
\begin{enumerate}
    \item 生成不同类型的测试数据:随机序列、有序序列、逆序序列、部分重复序列。
    \item 分别对自定义堆排序与 \texttt{std::sort\_heap()} 进行排序操作。
    \item 记录每种排序方法的时间,进行性能比较。
\end{enumerate}

\section{测试结果与对比}
测试结果如下:
\begin{table}[H]
    \centering
    \caption{不同排序算法的性能对比 (单位:ms)}
    \begin{tabular}{|c|c|c|}
        \hline
        数据类型 & 自定义堆排序 & STL 堆排序 \\
        \hline
        随机序列 & 1311 & 1261 \\
        \hline
        有序序列 & 448 & 457 \\
        \hline
        逆序序列 & 462 & 480 \\
        \hline
        部分重复序列 & 1151 & 1139 \\
        \hline
    \end{tabular}
\end{table}

\section{结论}
通过对比测试结果可以看出:
\begin{itemize}
    \item 自定义堆排序在性能上与 STL 堆排序相当,且在部分情况下性能略优。
\end{itemize}


\end{document}
