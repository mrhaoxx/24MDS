\documentclass[UTF8]{ctexart}
\usepackage{geometry, CJKutf8}
\geometry{margin=1.5cm, vmargin={0pt,1cm}}
\setlength{\topmargin}{-1cm}
\setlength{\paperheight}{29.7cm}
\setlength{\textheight}{25.3cm}

\usepackage{amsfonts}
\usepackage{amsmath}
\usepackage{amssymb}
\usepackage{amsthm}
\usepackage{enumerate}
\usepackage{graphicx}
\usepackage{multicol}
\usepackage{fancyhdr}
\usepackage{layout}
\usepackage{listings}
\usepackage{float, caption}
\usepackage{hyperref}

\lstset{
    basicstyle=\ttfamily,
    basewidth=0.5em
}

\newcommand{\dif}{\mathrm{d}}
\newcommand{\avg}[1]{\left\langle #1 \right\rangle}
\newcommand{\difFrac}[2]{\frac{\dif #1}{\dif #2}}
\newcommand{\pdfFrac}[2]{\frac{\partial #1}{\partial #2}}
\newcommand{\OFL}{\mathrm{OFL}}
\newcommand{\UFL}{\mathrm{UFL}}
\newcommand{\fl}{\mathrm{fl}}
\newcommand{\op}{\odot}
\newcommand{\Eabs}{E_{\mathrm{abs}}}
\newcommand{\Erel}{E_{\mathrm{rel}}}

\title{四则运算表达式求值程序设计与分析报告}
\author{作者:郝星星 \& ChatGPT}
\date{\today}

\begin{document}
\maketitle

\section{引言}
本报告介绍了一个四则运算表达式求值程序的设计思路和实现方法。该程序支持加减乘除以及括号的中缀表达式求值,并对非法输入进行严谨的判断和处理。用户可输入任意含有小数、科学计数法以及负数的中缀表达式,程序将输出计算结果或在检测到错误时输出“ILLEGAL”。

\section{需求和挑战}
在实现四则运算表达式求值时需要解决以下几个关键问题:
\begin{enumerate}
    \item \textbf{支持多类型输入元素}:  
    表达式中可能出现整数、小数、科学计数法的数值、加减乘除运算符及多层嵌套的括号。
    \item \textbf{错误检测与处理}:  
    需要对各种非法情况进行识别和处理,包括:
    \begin{itemize}
        \item 括号不匹配
        \item 连续运算符(如$1++2$)
        \item 表达式起始或结束处为运算符
        \item 除数为零
        \item 非法字符(如$1a+2$)
        \item 科学计数法格式不正确(如$2e+$)
    \end{itemize}
    对于非法表达式,程序统一输出“ILLEGAL”。
\end{enumerate}

\section{设计思路}
为实现通用且可靠的求值程序,本报告采用了标准的三步法:
\begin{enumerate}
    \item \textbf{分词(Tokenization)}:  
    将输入字符串解析为一个有序的记号序列(token list),包括数字(支持小数和科学计数法)、运算符和括号。  
    在此步骤对数字格式进行检查,并识别可作为负号的单目减号。

    \item \textbf{中缀转后缀(Shunting Yard算法)}:  
    将中缀表达式转换为后缀(逆波兰表示法)表达式。  
    运用优先级和栈结构处理运算符和括号,并在此过程中确保表达式结构合法,如检测括号配对和连续运算符。

    \item \textbf{后缀求值}:  
    使用栈对后缀表达式求值。当遇到运算符时,从栈中弹出对应数量的操作数进行计算,再将结果压回栈中。  
    在此步骤检测除零错误以及最终结果栈中元素个数是否正确。
\end{enumerate}

\section{结果分析}
\subsection{测试数据说明}
测试数据由若干行组成,每行一个表达式和预期结果,用空格分隔。如果预期结果为“ - ”,表示该表达式应被判断为非法。
例如:
\begin{itemize}
    \item $(1+2)*3$ 期望结果为 $9$,若程序输出9则表示通过测试。
    \item $1++2$ 期望结果为“ - ”,代表非法表达式,程序应输出“ILLEGAL”。
\end{itemize}

测试数据覆盖:
\begin{enumerate}
    \item 基本四则运算:如$1+2*3$
    \item 括号嵌套与优先级:如$((2+3)*5)/(7-2)$
    \item 负数与科学计数法:如$-1+2e2$
    \item 各类非法情形:如$1++2$、$1/0$、$3..5$、$((1+2)3)$
\end{enumerate}

\subsection{测试结果}
在对广泛的测试数据进行验证后,程序对于合法表达式能正确输出结果,对于非法表达式则能够及时输出“ILLEGAL”。例如:
\begin{itemize}
    \item 输入$(1+2)*3$,程序输出$9$,期望值为$9$,测试通过。
    \item 输入$1/0$,程序输出“ILLEGAL”,符合预期。
    \item 输入$1++2$,程序输出“ILLEGAL”,期望为“ - ”,测试通过。
    \item 输入$-1+2e2$,程序正确计算为$199$,符合期望值。
\end{itemize}

在对包含负数、小数、科学计数法、复杂嵌套括号以及非法字符和格式的测试用例进行验证后,程序表现出较高的鲁棒性和正确性。

\section{总结与展望}
本程序采用分词、中缀转后缀和后缀求值的经典策略,对四则运算表达式进行求值,并实现了对非法输入的严格检查。结果表明,该设计方案简单而有效,易于扩展。  
未来可考虑,支持更多数学函数(如$\sin$, $\cos$, $\log$等,也可加入包括求积分,求极限之类的函数。

\end{document}